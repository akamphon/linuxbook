\begin{thwbr}
\begin{thebibliography}{99}
\addcontentsline{toc}{chapter}{\numberline{}\bibname} 

\bibitem{fsf} Free Software Foundation\\
	อยู่ที่ \texttt{http://www.gnu.org}

\bibitem{license} Free Software Foundation, \newblock `Various Licenses and Comments about Them'\\
	อยู่ที่ \texttt{http://www.gnu.org/licenses/license-list.html}

\bibitem{freesoftware} Free Software Foundation, \newblock `The Free Software Definition'\\
	อยู่ที่ \texttt{http://www.gnu.org/philosophy/free-sw.html}

\bibitem{kernel} The Linux Kernel Archives, \\
	อยู่ที่ \texttt{http://www.kernel.org}

\bibitem{linuxnow} The Most Linux Complete Reference, \newblock `Starting Up With Linux'\\
	อยู่ที่ \texttt{http://new.linuxnow.com/tutorial/intro/starting.html}

\bibitem{debianhistory} `A Brief History of Debian'\\
	อยู่ที่ \texttt{http://www.debian.org/doc/manuals/project-history}

\bibitem{tlwg} `Thai Linux Working Group'\\
	อยู่ที่ \texttt{http://linux.thai.net}

\bibitem{osi} Open Source Initiative OSI, \newblock `The Open Source Definition'\\
	อยู่ที่ \texttt{http://www.opensource.org/docs/definition.php}

\bibitem{JustForFun} Linus Torvalds, David Diamod, \newblock `Just For Fun' \newblock HarperCollins \newblock 2001

\bibitem{magicgarden} Berny Goodheart and James Cox, \newblock `The Magic Garden Explained: The Internals of UNIX System V Release 4.' \newblock Prentice Hall \newblock 1994

\bibitem{newfrontier} Uresh Vahalia, \newblock `UNIX Internal: The New Frontiers' \newblock Prentice Hall \newblock 1996

\bibitem{osbook1} Andrew S. Tanenbaum, \newblock `Modern Operating Systems' \newblock Prentice Hall \newblock 1992

\bibitem{linuxhistory} วุฒิชัย~อัมพรอร่ามเวทย์, \newblock `\TeX{} ภาษาไทย'
    \newblock เป็นเว็บที่รวมโปรแกรมต่างๆที่จำเป็นกับการใช้งานภาษาไทยด้วย 
    \LaTeX{}\\ 
    อยู่ที่ \texttt{http://thaigate.rd.nacsis.ac.jp/files/ttex.html}


\bibitem{manual} Leslie Lamport.  \newblock \emph{{\LaTeX:} A Document
    Preparation System}.  \newblock Addison-Wesley, Reading,
  Massachusetts, second edition, 1994, ISBN~0-201-52983-1.
  
\bibitem{texbook} Donald~E. Knuth.  \newblock \textit{The \TeX{}book,}
  Volume~A of \textit{Computers and Typesetting}, Addison-Wesley,
  Reading, Massachusetts, second edition, 1984, ISBN~0-201-13448-9.

\bibitem{companion} Michel Goossens, Frank Mittelbach and Alexander
  Samarin.  \newblock \emph{The {\LaTeX} Companion}.  \newblock
  Addison-Wesley, Reading, Massachusetts, 1994, ISBN~0-201-54199-8.
 
\bibitem{local} สำหรับทุกๆระบบที่ลงโปรแกรม \LaTeX{} ควรจะมีคู่มือให้ 
  เรียกว่า \emph{\LaTeX{} Local Guide} ที่บอก ว่าระบบที่ใช้อยู่นั้นมีอะไรให้บ้าง
  อาจจะมีอยู่ในไฟล์ชื่อ \texttt{local.tex} แต่ก็ไม่แน่เสมอไป บางทีผู้ดูแลระบบ
  อาจจะไม่ได้ทำให้ไว้ก็เป็นได้ ถ้าเป็นอย่างนี้เห็นจะต้องหันไปพึ่งเซียน \LaTeX{} 
  รอบๆข้างดู
 
\bibitem{usrguide} \LaTeX3 Project Team.  \newblock \emph{\LaTeXe~for
    authors}.  \newblock  มีมาพร้อมกับชุดแจกจ่ายของ \LaTeXe{} ชื่อ
  \texttt{usrguide.tex}.  

\bibitem{clsguide} \LaTeX3 Project Team.  \newblock \emph{\LaTeXe~for
    Class and Package writers}.  \newblock มีมาพร้อมกับชุดแจกจ่ายของ 
    \LaTeXe{} ชื่อ \texttt{clsguide.tex}.

\bibitem{fntguide} \LaTeX3 Project Team.  \newblock \emph{\LaTeXe~Font
    selection}.  \newblock มีมาพร้อมกับชุดแจกจ่ายของ \LaTeXe{} ชื่อ
    \texttt{fntguide.tex}.

\bibitem{graphics} D.~P.~Carlisle.  \newblock \emph{Packages in the
    `graphics' bundle}.  \newblock มีมากับกลุ่มชุด `graphics' ชื่อ
  \texttt{grfguide.tex}, หาได้จากแหล่งเดียวกับที่คุณไปนำ \LaTeX{} มา

\bibitem{verbatim} Rainer~Sch\"opf, Bernd~Raichle, Chris~Rowley.  
\newblock \emph{A New Implementation of \LaTeX's verbatim
  Environments}.
 \newblock มีมากับกลุ่มชุด `tools' ชื่อ  \texttt{verbatim.dtx}, 
  หาได้จากแหล่งเดียวกับที่คุณไปนำ \LaTeX{} มา

\bibitem{catalogue} Graham~Williams.  \newblock \emph{The TeX
    Catalogue} รวบรวมรายชื่อแพคเกจต่างๆที่เกี่ยวข้องกับ \TeX{} และ 
    \LaTeX{} อย่างสมบูรณ์
  \newblock มีให้แล้วที่  \texttt{CTAN:/help/Catalogue/catalogue.html}
  
\bibitem{eps} Keith~Reckdahl.  \newblock \emph{Using EPS Graphics in
    \LaTeXe{} Documents} อธิบายทุกๆอย่างเกี่ยวกับไฟล์ชนิด EPS 
    และการนำมาใช้กับ \LaTeX{} ซึ่งบางทีออกจะมากกว่าที่คุณต้องการจะทราบ
    ด้วยซ้ำไป 
    \newblock มีให้แล้วที่ \texttt{CTAN:/info/epslatex.ps}
    
\bibitem{thtex} วุฒิชัย~อัมพรอร่ามเวทย์, \newblock `\TeX{} ภาษาไทย'
    \newblock เป็นเว็บที่รวมโปรแกรมต่างๆที่จำเป็นกับการใช้งานภาษาไทยด้วย 
    \LaTeX{}\\ 
    อยู่ที่ \texttt{http://thaigate.rd.nacsis.ac.jp/files/ttex.html}

\bibitem{thlatex} พูลลาภ~วีระธนาบุตร, \newblock `การใช้ภาษาไทยกับ \LaTeX{}' 
    \newblock เอกสารที่พูดถึงองค์ประกอบต่างๆในการใช้งาน \LaTeX{} 
    และการนำมาประยุกต์ใช้กับภาษาไทย\\
    อยู่ที่ \texttt{http://zzzthai.fedu.uec.ac.jp/linux/thailatex/}
    
\bibitem{typesetth} มานพ~วงศ์สายสุวรรณ, \newblock `รู้จักกับ Typesetter ภาษาไทย Thai\TeX{}' 
    \newblock ความเป็นมาเกี่ยวกับการใช้งานภาษาไทยกับ \TeX{}\\
    อยู่ที่ \texttt{http://thaigate.rd.nacsis.ac.jp/files/ttex.html}

\end{thebibliography}
\end{thwbr}
\wbrin