\begin{thwbr}
\chapter{การติดตั้งลินุกซ์}
ในบทนี้จะอธิบายการติดตั้งลินุกซ์สำหรับเครื่องคอมพิวเตอร์ส่วนบุคคลและใช้รูปประกอบเพื่อความชัดเจน. ระหว่างการติดตั้งผู้ใช้สามารถเลือกรายละเอียดปลีกย่อยซึ่งไม่จำเป็นต้องเหมือนกับคำอธิบายก็ได้.


ขั้นตอนการติดตั้งลินุกซ์โดยทั่วไปมีขั้นตอนดังนี้.
\begin{enumerate}
\item ดาวน์โหลดซีดีในรูปของดิสก์อิมเมจ
\item เขียนซีดีรอม และบูตเครื่องคอมพิวเตอร์จากซีดีรอม
\item ติดตั้งระบบปฏิบัติการลินุกซ์ด้วยโปรแกรมติดตั้งของดิสทริบิวชันที่ใช้
\item ปรับแต่งและอัปเดทซอฟต์แวร์
\end{enumerate}

\section{ดาวน์โหลดซีดี}
การติดตั้งลินุกซ์โดยทั่วไปมักจะใช้ซีดีรอมเป็นสื่อในการติดตั้ง. ผู้ใช้สามารถซื้อหรือจะดาวน์โหลดด้วยตัวเองจากเว็บไซด์ของดิสทริบิวชันต่างๆ.

ซีดีที่ดาว์นโหลดมาจะอยู่ในรูปของไฟล์ที่เรียกว่า\emph{ดิสก์อิมเมจ (disk image)} และมักจะมีส่วนขยายชื่อไฟล์เป็น \cmd{.iso}. %
\mymemo{ส่วนขยายชื่อไฟล์ \cmd{.iso} เป็นแค่ชื่อที่ให้คนอ่านแล้วเข้าใจ. จริงๆแล้วไม่จำเป็นต้องเป็น \cmd{.iso}, บ้างก็ใช้ส่วนขยายชื่อไฟล์เป็น \cmd{.raw}, {\latintext\tt .img} หรือไม่ใส่ส่วนขยายชื่อไฟล์.}%
ตัวอย่างเช่นแผ่นซีดีติดตั้งแผ่นแรกของ Debian 3.1

ลินุกซ์ทะเลรุ่น 5.5 มีชื่อ samila-5.5-i386-cd1.iso. ผู้ใช้ต้องดาวน์โหลดไฟล์นี้เพื่อนำมาเขียนซีดีรอม. ในกรณีที่แผ่นซีดีรอมติดตั้งมีมากกว่าหนึ่งแผ่น, ผู้ใช้ต้องดาวน์โหลดมาทั้งหมด. บางกรณีอาจจะมีรหัสต้นฉบับเป็นดิสก์อิมเมจให้ดาว์โหลดด้วย, แต่ไม่จำเป็นสำหรับการติดตั้ง. 

บางดิสทริบิวชันอาจจะมีซีดีหลายแผ่นแบ่งตามสถาปัตยกรรมของคอมพิวเตอร์ที่ต้องการใช้. ผู้อ่านควรจะอ่านเอกสารกำกับของดิสทริบิวชันนั้นๆให้ละเอียดก่อนดาวน์โหลด.

\subsection{การตรวจสอบดิสก์อิมเมจ}
ไฟล์ดิสก์อิมเมจที่ดาวน์โหลดมามักจะมีขนาดใหญ่ประมาณ 700 MB เพื่อเขียนลงซีดีรอมหนึ่งแผ่น. การดาวน์โหลดไฟล์ใหญ่แบบนี้อาจจะเกิดความผิดพลาดในการโอนถ่ายข้อมูล, ซึ่งบางครั้งนำไฟล์ที่ดาวน์โหลดไปเขียนซีดีแล้วใช้งานไม่ได้. ในบางกรณีที่ผู้ใช้ไม่ได้ดาวน์โหลดไฟล์จากเว็บไซด์ของดิสทริบิวชันอย่างเป็นทางการ, อาจจะมีผู้ไม่ประสงค์ดีแอบใส่โปรแกรมที่ไม่พึงประสงค์ลงไปในไฟล์ติดตั้งด้วย. ด้วยเหตุผลเหล่านี้เองผู้ใช้จึงควรตรวจสอบไฟล์ดิสก์อิมเมจที่ดาวน์โหลดมาก่อนใช้งาน.

ผู้ใช้สามารถตรวจสอบไฟล์ที่ดาวน์โหลดมาด้วยโปรแกรม \cmd{md5sum}. %
%
\begin{MyVerbatim}
$ \cin{md5sum samila-5.5-i386-cd1.iso}
897e058764206b32bf9c47d4d3d51c7e  samila-5.5-i386-cd1.iso
\end{MyVerbatim}
โปรแกรม \cmd{md5sum} จะแสดงค่า\emph{เช็คซัม (checksum)} ของไฟล์ที่ต้องการตรวจสอบ. เมื่อได้ค่าเช็คซัมแล้วก็นำไปเทียบกับค่าเช็คซัมที่ดิสทริบิวชันแสดงไว้ว่าตรงกันหรือไม่. ถ้าไม่ตรงกันแสดงว่าไฟล์ที่ดาวน์โหลดมามีข้อมูลไม่ตรงกับต้นฉบับจริง. ดิสทริบิวชันมักจะเตรียมไฟล์รายการค่าเช็คซัมกับชื่อไฟล์ไว้ให้ดาวน์โหลดด้วยเพื่อที่ผู้ใช้จะได้ตรวจสอบได้ว่าไฟล์ที่ดาวน์โหลดไปถูกต้องหรือไม่.

สำหรับผู้ที่ใช้วินโดวส์และต้องการตรวจค่าเช็คซัม, ให้ใช้โปรแกรม MD5summer \cite{md5summer} ซึ่งเป็นซอฟต์แวร์เสรี.

\section{การเขียนซีดี}
โปรแกรมที่ใช้เขียนซีดีบนลินุกซ์มีหลายโปรแกรมให้เลือกใช้เช่น \cmd{xcdroast}, \cmd{gtoaster} แต่โปรแกรมเหล่านี้ต่างก็เป็น frontend ของโปรแกรม (คำสั่ง) \cmd{cdrecord} ทั้งนั้น. เมื่อดาวน์โหลดัและตรวจสอบไฟล์ที่ดาวน์โหลดแล้ว, ให้ใช้คำสั่ง \cmd{cdrecord}. 

\myexplanation{cdrecord}{\cmd{dev=0,0,0} เป็นการระบุดีไวส์ซีดี. \cmd{-eject} ให้คอมพิวเตอร์ดีดตัวซีดีออกจากไดรว์เมื่อเขียนซีดีเสร็จ. และ \cmd{speed=4} เป็นการใช้ความเร็ว 4 เท่า.}
\begin{MyVerbatim}
# \cin{cdrecord dev=0,0,0 -eject speed=4 samila-5.5-i386-cd1.iso}
\end{MyVerbatim} 

ผู้ที่ใช้วินโดวส์สามารถใช้โปรแกรมเขียนไฟล์ \cmd{.iso} ลงซีดีได้เช่น Nero, Easy CD Creator ฯลฯ.
\section{การติดตั้งลินุกซ์}
ก่อนอื่นให้ตรวจสอบ BIOS ของคอมพิวเตอร์ให้เครื่องคอมพิวเตอร์บูตเครื่องจากซีดีได้. จากนั้นให้ใส่ซีดีแล้วเปิดเครื่องใหม่, ถ้าไม่มีข้อผิดพลาดใดๆก็จะสามารถบูตจากแผ่นซีดีได้.

\subsection{ลินุกซ์ทะเล}

\vfill
\subsubsection*{เข้าสู่โปรแกรมติดตั้ง}\mymemo{ผู้อ่านสามารถเลือกวิธีการติดตั้ง (อินเทอร์เฟส) ได้สองแบบคือแบบกราฟฟิกและแบบเท็กซ์โหมด. ในที่นี้จะอธิบายวิธีการติดตั้งแบบกราฟฟิก. หากต้องการติดตั้งในระบบเท็กซ์โหมด, ให้พิมพ์ \cmd{linux text} แล้วกดคีย์ Enter. การติดตั้งแบบกราฟฟิกให้กดคีย์ Enter เพื่อดำเนินการต่อไป. 

นอกจากการติดตั้งแล้วผู้อ่านสามารถใช้แผ่นซีดีเป็นแผ่นบูตคอมพิวเตอร์แก้ไขปัญหาเวลาเครื่องบูตไม่ได้. ในกรณีนี้เรียกว่าการบูตแบบ Rescue. นอกจากนั้นผู้อ่านสามารถเลือกวิธีการติดตั้งจากเน็ตเวิร์คได้ด้วย.}

\bigskip
\includegraphics[scale=.45]{TLE_samila_001.eps}

\vfill
\subsubsection*{หน้าจอต้อนรับของโปรแกรมติดตั้ง}\mymemo{ด้านซ้ายมือของหน้าจอจะเป็นหน้าต่างช่วยเหลือ, ช่วยอธิบายขั้นตอนการติดตั้ง. ในช่วงนี้ผู้อ่านสามารถกดปุ่มอ่าน\emph{รีลีสโน้ต (release note)} ซึ่งจะแนะนำความสามารถใหม่ในลินุกซ์ทะเล. ให้กดปุ่ม ``ถัดไป'' เพื่อดำเนินการต่อไป.}

\bigskip
\includegraphics[scale=.38]{TLE_samila_002.eps}
\vfill

\clearpage
\vfill
\subsubsection*{เลือกแป้นพิมพ์}\mymemo{แป้นพิมพ์ภาษาไทยที่สามารถเลือกใช้ได้แก่ แป้นพิมพ์เกษมณี, แป้นพิมพ์ปัตโชติ และ แป้นพิมพ์ TIS-820.2538. 

แป้นพิมพ์โดยปริยายได้แก่แป้นพิมพ์เกษมณี. แป้นพิมพ์ TIS-820.2538 เป็นแป้นพิมพ์ที่ขยายความสามารถแป้นพิมพ์เกษมณีสามารถพิมพ์ตัวอักษรที่แป้นพิมพ์เกษมณีพิมพ์ไม่ได้เช่น \textfongmun{} \textkhomut.}
\includegraphics[scale=.38]{TLE_samila_003.eps}
\vfill

\subsubsection*{เลือกประเภทเมาส์}\mymemo{``จำลองเป็นแบบ 3 ปุ่ม'' มีไว้สำหรับเมาส์ที่มี 2 ปุ่ม. เวลากดเมาส์ 2 ปุ่มพร้อมๆกันจะจำลองให้เหมือนการกดเมาส์ปุ่มตรงกลางของเมาส์ที่มี 3 ปุ่ม.}
\includegraphics[scale=.38]{TLE_samila_004.eps}
\vfill
\clearpage

\vfill
\subsubsection*{เลือกหน้าจอแสดงผล}\mymemo{โดยปรกติแล้วโปรแกรมติดตั้งจะรับรู้ประเภทของจอภาพโดยอัตโนมัติ. ถ้าโปรแกรมติดตั้งไม่สามารถรับรู้ประเภทของจอภาพโดยอัตโนมัติ, ให้ดูหนังสือคู่มือของจอภาพประกอบแล้วเลือกจอภาพแบบ Generic ที่เห็นว่าเหมาะสม. ในกรณีนี้อาจดูคู่มือของจอภาพแล้วใส่ค่าสัญญาณตามแนวนอนหรือสัญญาณตามแนวตั้งเองก็ได้.}
\includegraphics[scale=.38]{TLE_samila_005.eps}
\vfill

\subsubsection*{เลือกชนิดการติดตั้ง}\mymemo{เลือกชนิดการติดตั้งตามที่ต้องการได้แก่ เดสก์ทอป, เวิร์กสเตชัน, เซิร์ฟเวอร์ หรือ เลือกแพ็กเกจเอง. สำหรับผู้ใช้ใหม่อาจจะเลือกการติดตั้งแบบเดสก์ทอปดูก่อนก็ได้.}
\includegraphics[scale=.38]{TLE_samila_006.eps}
\vfill
\clearpage

\vfill
\subsubsection*{เลือกวิธีแบ่งพาร์ทิชัน}\mymemo{ช่วงนี้ผู้อ่านสามารถเลือกวิธีแบ่งพาร์ทิชันได้สองแบบคือแบบอัตโนมัติหรือใช้ Disk Druid แบ่งพาร์ทิชัน. ในที่นี้จะอธิบายวิธีใช้ Disk Druid แบ่งพาร์ทิชัน.}
\includegraphics[scale=.38]{TLE_samila_007_b01.eps}
\vfill

\subsubsection*{เลือกฮาร์ดดิสก์}\mymemo{ในตัวอย่างเป็นการแบ่งพาร์ทิชันของฮาร์ดดิสก์ใหม่. จะเห็นเป็นพื้นที่ว่างทั้งฮาร์ดดิสก์. ต่อไปนี้จะแบ่งพาร์ทิชันเป็นสองส่วน. ส่วนแรกสำหรับ \cmd{/} (รูทไดเรกทอรี) และอีกส่วนสำหรับใช้เป็น swap.

\bigskip
คลิ้กพื้นที่ว่างแล้วกดปุ่ม ``สร้างใหม่'' เพื่อสร้างพาร์ทิชันใหม่.}
\includegraphics[scale=.38]{TLE_samila_007_b02.eps}
\vfill
\clearpage

\vfill
\subsubsection*{สร้างรูทพาร์ทิชัน}\mymemo{ในตัวอย่างจะสร้างรูทพาร์ทิชันมีขนาดประมาณ 3750 MB และให้ส่วนที่เหลือเป็น swap.

\bigskip
เลือกตำแหน่งเมานท์ของรูทพาร์ทิชันเป็น \cmd{/} และให้ชนิดของระบบไฟล์เป็น ext3. เซ็ตให้ขนาดของพาร์ทิชันคงที่แล้วกดปุ่ม ``ตกลง''.

\bigskip
สำหรับผู้ใช้ที่ชำนาญแล้วอาจจะแบ่งพาร์ทิชันสำหรับ \cmd{/boot}, \cmd{/home}, \cmd{/var} ฯลฯ แยกออกจากรูทไดเรกทอรีก็ได้.}
\includegraphics[scale=.38]{TLE_samila_007_b03.eps}
\vfill

\subsubsection*{สภาพหลังจาการสร้างรูทพาร์ทิชัน}\mymemo{รูปประกอบแสดงสภาพของฮาร์ดดิสก์หลังจากสร้างพาร์ทิชันสำหรับรูทไดเรกทอรีเรียบร้อยแล้ว. จะเห็นว่ามีเนื้อที่ว่างอยู่และเราจะใช้พื้นที่ว่างนี้สร้าง swap พาร์ทิชันต่อไป.}
\includegraphics[scale=.38]{TLE_samila_007_b04.eps}
\vfill
\clearpage

\vfill
\subsubsection*{สร้างพาร์ทิชันสำหรับ swap}\mymemo{ให้เลือกชนิดของระบบไฟล์เป็น swap. แล้วกดปุ่ม ``ตกลง''.}
\includegraphics[scale=.38]{TLE_samila_007_b05.eps}
\vfill

\subsubsection*{สภาพหลังจากสร้างพาร์ทิชันเสร็จ}\mymemo{รูปประกอบแสดงสภาพของฮาร์ดดิสก์หลังจากสร้างพาร์ทิชันทั้งหมดเรียบร้อยแล้ว. ถ้าต้องการเปลี่ยนแปลงแก้ไขสามารถกดปุ่มย้อนกลับเริ่มแบ่งพาร์ทิชันใหม่ได้.}
\includegraphics[scale=.38]{TLE_samila_007_b06.eps}
\vfill
\clearpage


\vfill
\subsubsection*{}\mymemo{}
\includegraphics[scale=.38]{TLE_samila_008.eps}
\vfill

\subsubsection*{}\mymemo{}
\includegraphics[scale=.38]{TLE_samila_008_01.eps}
\vfill
\clearpage

\vfill
\subsubsection*{}\mymemo{}
\includegraphics[scale=.38]{TLE_samila_009.eps}
\vfill

\subsubsection*{}\mymemo{}
\includegraphics[scale=.38]{TLE_samila_010.eps}
\vfill
\clearpage



\end{thwbr}
\wbrin
